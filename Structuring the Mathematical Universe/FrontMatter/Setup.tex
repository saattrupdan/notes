% Document definition
\documentclass[a4paper,12pt,openany]{book}

% Packages
\usepackage{fullpage}
\usepackage{amsmath}
\usepackage{amsfonts}
\usepackage{amssymb}
\usepackage{amsthm}
\usepackage{mathrsfs}
\usepackage{graphicx}
\usepackage[utf8x]{inputenc}
\usepackage{fancyhdr}
\usepackage{sectsty}
%\usepackage{hyperref}

% Font size
\sectionfont{\large}
\chapterfont{\LARGE}

% No indent
\setlength{\parindent}{0in}

% Theorem environment
\theoremstyle{plain}
\newtheorem{theorem}[subsection]{Theorem}
\newtheorem{proposition}[subsection]{Proposition}
\newtheorem{lemma}[subsection]{Lemma}
\newtheorem{corollary}[subsection]{Corollary}
\theoremstyle{definition}
\newtheorem{definition}[subsection]{Definition}
\newtheorem{axiom}[subsection]{Axiom}
\newtheorem{example}[subsection]{Example}
\theoremstyle{remark}
\newtheorem{remark}[subsection]{Remark}

% User-defined commands
\renewcommand{\labelenumi}{(\roman{enumi}) }
\newcommand{\biglor}{\bigvee}
\newcommand{\bigland}{\bigwedge}
\newcommand{\ip}[2]{\left<#1,#2\right>}
\newcommand{\src}[1]{\left[#1\right]}
\newcommand{\norm}[1]{\left|\left|#1\right|\right|}
\renewcommand{\b}[1]{{\bf #1}}
\newcommand{\theo}[2][]{\begin{theorem}[#1]#2\end{theorem}}
\newcommand{\prop}[2][]{\begin{proposition}[#1]#2\end{proposition}}
\newcommand{\lemm}[2][]{\begin{lemma}[#1]#2\end{lemma}}
\newcommand{\coro}[2][]{\begin{corollary}[#1]#2\end{corollary}}
\newcommand{\defi}[2][]{\begin{definition}[#1]#2\end{definition}}
\newcommand{\axio}[2][]{\begin{axiom}[#1]#2\\\end{axiom}}
\newcommand{\exam}[2][]{\begin{example}[#1]#2\\\end{example}}
\newcommand{\rema}[2][]{\begin{remark}[#1]#2\end{remark}}
\renewcommand{\qed}{\hfill\blacksquare}
\newcommand{\qedeq}{\tag*{$\blacksquare$}}
\newcommand{\mrep}[3]{_{\mathscr #3}#1_{\mathscr #2}}
\newcommand{\lr}{\Leftrightarrow}
\newcommand{\sgn}{\text{sgn}}
\newcommand{\lcm}{\text{lcm}}
\newcommand{\vc}[1]{\underline{#1}}
\newcommand{\vto}[2]{\begin{pmatrix}#1\\#2\end{pmatrix}}
\newcommand{\mx}[1]{\underline{\underline{#1}}}
\newcommand{\mto}[4]{\begin{pmatrix} #1 & #2 \\ #3 & #4\end{pmatrix}}
\newcommand{\mtre}[9]{\begin{pmatrix} #1 & #2 & #3 \\ #4 & #5 & #6 \\ #7 & #8 & #9\end{pmatrix}}
\newcommand{\eq}[1]{\begin{align*} #1\\ \end{align*}}
\newcommand{\pic}[1]{\begin{center}\includegraphics[scale=.5]{#1.png}\\\end{center}}
\renewcommand{\subset}{\subseteq}
\newcommand{\bra}[1]{\left<#1\right>}