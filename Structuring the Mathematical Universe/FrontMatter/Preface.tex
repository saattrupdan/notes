\chapter*{Preface}
\thispagestyle{fancy}
This booklet is not designed to teach the reader about a subject from start to finish. It is not intended as an article, providing new results. It is designed for people who seek an overview - people who have learned about various mathematical subjects and want to know how these fit together. It also doubles as an encyclopaedia through a wide range of examples of mathematical structures.\\

To make it as easy as possible to look up definitions and links, I have refrained myself from writing unnecessary explanatory text and have tried to be as concise as possible while still giving the necessary information for understanding the structure at hand.\\

The map, which illustrates all the links described in this booklet, has been categorized within the same structure as this booklet's chapters. Furthermore, the arrows in the map indicate if a structure implies another structure. If the arrow is based on a definition, it is coloured black. If it is based on a theorem, remark or anything else not a definition, it is coloured green.\\

Substructures are used in the usual sense, being that algebraic substructures are closed under their operations, relational substructures have their relation restricted to the underlying set and pavings of substructures are the intersection between the elements of the paving and the underlying subset. I assume ETCS; an (informal) overview of the axioms is given in the first chapter.\\

My notation used is pretty standard. I use the bold notation $\b{A}$ to denote a structure, and the corresponding non-bold letter $A$ to denote the underlying set. I use $|A|$ to denote the cardinality of $A$, $\omega$ as the first infinite ordinal and $\aleph_0=|\omega|$. Structures will be denoted with angled brackets $\langle A,f,g,h,\hdots\rangle$, where $A$ is a set and $\{f,g,h,\hdots\}$ is the language of the structure.\\

New versions will be available at
\begin{center}
\texttt{bit.ly/mathuniverse}.
\end{center}