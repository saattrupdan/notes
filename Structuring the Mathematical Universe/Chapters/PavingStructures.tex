\chapter{Paving structures}
\thispagestyle{fancy}

\defi{
A \textit{family of sets} indexed by the class $I\neq \emptyset$ is a class of sets\\
$(A_i)_{i\in I}:=\{A_i\ |\ i\in I\}$. For such a family, its union and intersection are defined to be respectively the classes
\eq{
&\bigcup_{i\in I}A_i:=\{a\ |\ \exists i\in I: a\in A_i\}\\
&\bigcap_{i\in I}A_i:=\{a\ |\ \forall i\in I: a\in A_i\}
}
}

\section{Topological spaces}
\defi{
\label{topologicalspacedef}
A \textit{topology} on a set $A$ is a family of subsets of $A$ denoted $\mathcal{O}$, which satisfies
\begin{enumerate}
\item[($T_1$)] $\emptyset,A\in\mathcal{O}$.
\item[($T_2$)] $\forall G_1,\hdots,G_n\in\mathcal{O}:G_1\cap\hdots\cap G_n\in\mathcal{O}$
\item[($T_3$)] $(A_i)_{i\in I}\in\mathcal{O}\Rightarrow\bigcup_{i\in I} A_i\in\mathcal{O}$.\\
\end{enumerate}
}

\defi{
A \textit{topological space} $\langle A,\mathcal{O}\rangle$ is a set $A$ equipped with a topology $\mathcal{O}$.\\
}

\defi{
Let $A$ be a set. Then a \textit{basis} for a topology $\mathcal{O}$ on $A$ is a collection $\mathcal{B}$ of subsets of $A$ (called basis elements), such that
\begin{enumerate}
\item $\forall a\in A\exists B\in\mathcal{B}: x\in B$
\item $\forall a\in A\forall B_1,B_2\in\mathcal{B}: a\in B_1\cap B_2 \Rightarrow \exists B_3\in\mathcal{B}: x\in B_3\subset B_1\cap B_2$\\
\end{enumerate}
}

\pagebreak
\lemm{
\label{topbasisisunionsLemma}
Let $\langle A,\mathcal{O}\rangle$ be a topological space and let $\mathcal{B}$ be a basis for $\mathcal{O}$. Then $\mathcal{O}$ equals the collection of all unions of elements of $\mathcal{B}$.
}
\textit{Proof.}
Given a collection of elements of $\mathcal{B}$, they are also elements of $\mathcal{O}$. Because $\mathcal{O}$ is a topology, their union is in $\mathcal{O}$. Conversely, given $U\in\mathcal{O}$, choose for each $a\in U$ an element $B_a\in\mathcal{B}$ such that $a\in B_a\subset U$. Then $U=\bigcup_{a\in U} B_a$, so $U$ equals a union of elements of $\mathcal{B}$.$\qed$\\

\lemm{
\label{collectionopenistopLemma}
Let $\langle A,\mathcal{O}\rangle$ be a topological space. If $\mathcal{C}\subset\mathcal{O}$ satisfies
\eq{
\forall S\in\mathcal{O},a\in S\exists C\in\mathcal{C}: a\in C\subset S,
}

then $\mathcal{C}$ is a basis for $\mathcal{O}$.
}
\textit{Proof.}
We must show that $\mathcal{C}$ is a basis. The first condition for a basis is shown: Given $a\in A$, since $A\in\mathcal{O}$, there is by assumption a $C\in\mathcal{C}$ such that $x\in C\subset A$. To check the second condition, let $a\in C_1\cap C_2$, where $C_1,C_2\in\mathcal{C}$. Since $C_1,C_2\in\mathcal{O}$, then $C_1\cap C_2\in\mathcal{O}$.Therefore, there exists by assumption an element $C_3\in\mathcal{C}$ such that $x\in C_3\subset C_1\cap C_2$.\\

We must now show that the topology $\mathcal{O}'$ generated by $\mathcal{C}$ equals $\mathcal{O}$. First, note that if $U\in\mathcal{O}$ and if $x\in U$, then there is by assumption an element $C\in\mathcal{C}$ such that $x\in C\subset U$. It follows that $U\in\mathcal{O}'$ by definition. Conversely, if $W\in\mathcal{O}'$, then $W$ is a union of elements of $\mathcal{C}$, by Lemma \ref{topbasisisunionsLemma}. Since each element of $\mathcal{C}$ belongs to $\mathcal{O}$ and $\mathcal{O}$ is a topology, $W\in\mathcal{O}$.$\qed$.\\

\theo{
\label{metricspaceistopspace}
Every metric space $\langle A,d\rangle$ induces a topological space $\langle A,d,\mathcal{O}\rangle$.
}
\textit{Proof.}
Let $\langle A,d\rangle$ be a metric space and denote $B_d(a,\varepsilon):=\{b\in A\mid d(a,b)<\varepsilon\}$ the $\varepsilon$-ball centered on $b\in A$. Now define $\mathcal{G}$ as the family of all $\varepsilon$-balls $B_d(x,\varepsilon)$ for $a\in A$ and $\varepsilon>0$. It is clear that $\mathcal{G}$ satisfies the condition given in Lemma \ref{collectionopenistopLemma}, meaning that $\mathcal{G}$ is a basis for a topological space. Then by Lemma \ref{topbasisisunionsLemma}, the collection of all unions of $\mathcal{G}$ is a topology $\mathcal{O}$ on $A$. Thus is $\langle A,d,\mathcal{O}\rangle$ an induced topological space.$\qed$\\

\theo{
\label{topspaceismonoid}
Every topological space $\langle A,\mathcal{O}\rangle$ induces monoids $\langle A,\mathcal{O},\cup,\emptyset\rangle$ and $\langle A,\mathcal{O},\cap,A\rangle$.
}
\textit{Proof.}
Let $\langle A, \mathcal{O}\rangle$ be a topological space. It has to be shown that the induced $\langle A,\mathcal{O},\cup\rangle$ is a semigroup and has an identity element $e$, since the proof for inducing $\langle A,\mathcal{O},\cap\rangle$ follows analogously. To show that it is a semigroup, it is firstly noted that it is a magma, since $\cup$ is a binary operation by $(T_3)$. It has to be shown that $\cup$ is associative:
\eq{
a\in A\cup(B\cup C) &\lr a\in A\lor (a\in B\lor a\in C)\\
&\lr (a\in A\lor a\in B)\lor a\in C\\
&\lr a\in (A\cup B)\cup C
}

Thus is $\langle A,\mathcal{O},\cup\rangle$ a semigroup, and it has to be shown that it has an identity. Let $S\in\mathcal{O}$ and consider the empty set $\emptyset$, which exists by $(T_1)$:
\eq{
&\emptyset\cup S=\{a\in A\mid a\in S \lor a\in \emptyset\}\stackrel{\ref{emptysetdef}}{=}\{a\in A\mid a\in S\}= S\\
&S\cup \emptyset=\{a\in A\mid a\in \emptyset \lor a\in S\}\stackrel{\ref{emptysetdef}}{=}\{a\in A\mid a\in S\}= S
}

Thus is $e=\emptyset\in\mathcal{O}$, making $\langle A,\mathcal{O},\cup,\emptyset\rangle$ a monoid. Similarly it can be shown that $\langle A,\mathcal{O},\cap,A\rangle$ is a monoid as well by $(T_1)$ and $(T_2)$.$\qed$\\

\theo{
Every topological space $\langle A,\mathcal{O}\rangle$ induces a complete JID lattice $\langle A,\mathcal{O},\cup,\cap,\emptyset,A\rangle$.
}
\textit{Proof.}
Let $\langle A,\mathcal{O}\rangle$ be a topological space. It was shown in Theorem \ref{topspaceismonoid} that both $\langle A,\mathcal{O}, \cup, \emptyset\rangle$ and $\langle A,\mathcal{O},\cap,A\rangle$ were monoids. Since $\cup$ and $\cap$ are well known to be both idempotent and commutative, it has to be shown that $\langle A,\mathcal{O},\cup,\cap,\emptyset,A\rangle$ satisfies the absorption laws, distributive laws and completeness. The absorption laws and distributive laws follows analogously from the proof in Theorem \ref{measurablespaceisboolalgebra}, so it has to be shown that every set $S_i\subset A$ has both a supremum $\biglor S_i:=\bigcup S_i$ and an infimum $\bigland S_i:=\bigcap S_i$.\\

By $(T_2)$, $\b{A}$ is closed under arbitrary unions, meaning $\biglor S_i$ exists. Since topologies aren't closed under arbitrary intersections, the infimum can instead be constructed by taking the interior of arbitrary intersections. Since interiors are always open, the infimum exists. Moreover, the join-infinite distribution law is shown. Let $a\in A,S_0\in\mathcal{O},(S_i)_{i\in I}\subset\mathcal{O}$. Then
\eq{
a\in S_0\cap\bigcup_{i\in I}S_i&\lr (a\in S_0)\land(a\in \bigcup_{i\in I}S_i\lr (a\in S_0)\land(\exists i\in I: a\in S_i)\\
&\lr \exists i\in I: (a\in S_i)\land(a\in S_0)\lr a\in\bigcup_{i\in I}(S_i\cap S_0)\qedeq.
}

\section{First-countable spaces}
\defi{
Let $\langle A,\mathcal{O}\rangle$ be a topological space and let $a\in A$. Then a \textit{neighbourhood} of $a$ is a subset $S\subset A$ which satisfies that
\eq{
\exists O\in\mathcal{O}: a\in S\subset O.
}
}

\defi{
A \textit{first-countable space} $\langle A,\mathcal{O}\rangle$ is a topological space with a countable neighbourhood basis - that is, for every $a\in A$ there exists a sequence of open neighbourhoods $(S_i)_{i\in\mathbb{N}}$ of $a$, such that
\eq{
\forall U\in\mathcal{O}\exists i\in\mathbb{N}: x\in U\Rightarrow x\in S_i\subset U.
}
}

\theo{
\label{metricspaceisfirstcountspace}
Every metric space $\langle A,d\rangle$ induces a first-countable space $\langle A,d,\mathcal{O}\rangle$.
}
\textit{Proof.}
Let $\langle A,d\rangle$ be a metric space and $a\in A$. Since every metric space induces a topological space by Theorem \ref{metricspaceistopspace}, it has to be shown that $\b{A}$ has a countable local basis. Define such a candidate $\mathcal{B}:=\{B_{1/n}(a)\mid n\in\mathbb{N}\}$, where $B_\varepsilon(a)$ denotes the open $\varepsilon$-ball in $A$ with center in $a$ and radius $\varepsilon$.\\

First of all, since a surjection $f:\mathbb{N}\to\mathcal{B}$ can be created, $\mathcal{B}$ is countable. It is clear that every element $b\in\mathcal{B}$ is an open neighbourhood of $a$, since the balls are open with $a$ as center. Now let $U\subset A$ be an open neighbourhood of $a$. By definition of open sets, an open ball $B_\varepsilon(a)\subset U$. By the Archimedean principle, there exist an $n\in\mathbb{N}$ such that $n>\frac{1}{\varepsilon}$ which implies $\varepsilon>\frac{1}{\varepsilon}$, meaning $B_{1/n}(a)\subset B_\varepsilon(a) \subset U$. This proves that $\mathcal{B}$ is a countable local basis for $A$.$\qed$\\

\section{Separable spaces}
\defi{
A \textit{separable space} $\langle A,\mathcal{O}\rangle$ is a topological space in which there exist a countable dense subset - that is, a sequence $(a_n)_{n\in\mathbb{N}}\subset A$, which satisfies
\eq{
\forall U\in\mathcal{O}\exists i\in\mathbb{N}: a_i\in U.
}
}

\section{Second-countable spaces}
\defi{
A \textit{second-countable space} $\langle A,\mathcal{O},\mathcal{B}\rangle$ is a topological space $\langle A,\mathcal{O}\rangle$ with a countable collection of sets $\mathcal{B}=(B_i)_{i\in\mathbb{N}}\subset\mathcal{O}$, called a countable basis, such that
\eq{
\forall V\in\mathcal{O}\exists (B_k)_{k\in\mathbb{N}}\subset\mathcal{B}:V=\bigcup_{k\in\mathbb{N}}B_k.
}
}

\theo{
Every second-countable space reduces to both a first-countable space and a separable space.
}
\textit{Proof.}
Let $\langle A,\mathcal{O},\mathcal{B}\rangle$ be a second-countable space. For each $a\in A$, the set $\{U\in \mathcal{B}\mid a\in U\}$ is a countable neighbourhood basis. Therefore it is a first-countable space. Let $S$ consist of one point from each member of $\mathcal{B}$. Then $S$ is a countable dense subset of $A$. Therefore it is a separable space.$\qed$\\

\theo{
\label{sepmetricspaceissecondcount}
Every separable metric space $\langle A,d,\mathcal{O}\rangle$ induces a second-countable space $\langle A,d,\mathcal{O},\mathcal{B}\rangle$.
}
\textit{Proof.}
Let $\langle A,d\rangle$ be a metric space with a countable dense subset $S$. We will show that the set $\mathcal{B}$ of open balls with rational radii around points in $S$ form a countable basis for $A$. Consider the basic open set $B_r(a)$, for some 
$a\in A$ and $r>0$. Let $b\in B_r(a)$. There is an $s>0$ such that $B_s(b)\subset B_r(x)$. Let $c\in S\cap B_{s/3}(b)$ and let $t\in (s/3,2s/3)\cap\mathbb{Q}$. Then $b\in B_t(c)$, and it follows from the triangle inequality that $B_t(c)\subset B_s(b)\subset B_r(a)$. So we have shown that every point in $B_r(a)$ is inside a set in $\mathcal{B}$ that is contained in $B_r(a)$. Therefore $B_r(a)$ is a union of members of $\mathcal{B}$, and therefore $\mathcal{B}$ is a basis. Since $\mathcal{B}$ is in one-to-one correspondance with the product $S\times((0,\omega)\cap\mathbb{Q})$ of countable sets, $\mathcal{B}$ is countable.$\qed$\\

\theo{
Every Euclidean space $\b{A}$ induces a second-countable space $\langle \b{A},\mathcal{O},\mathcal{B}\rangle$.
}
\textit{Proof.}
Let $\b{A}$ be a Euclidean space. By definition, $\b{A}$ is a Hilbert space and an inner product space. Then by Theorem \ref{innerprodisnorm}, $\b{A}$ induces a normed space. Then by Theorem \ref{normedspaceismetricspace}, $\b{A}$ induces a metric space. By Theorem \ref{sepmetricspaceissecondcount}, it thus suffices to show that $\b{A}$ is separable, ie. that it has a countable dense subset. Consider $\mathbb{Q}^n$. It is known that $\mathbb{Q}^n$ is countable for all $n\in\mathbb{N}$. Furthermore, it is clear that $\forall n\in\mathbb{N}: \mathbb{Q}^n\subset\mathbb{R}^n$. Lastly, it is also known that $\mathbb{Q}$ is dense in $\mathbb{R}$, making $\mathbb{Q}^n$ dense in $\mathbb{R}^n$ by applying it to each coordinate. Thus by Theorem \ref{sepmetricspaceissecondcount}, $\b{A}$ is a second-countable space.$\qed$\\

\section{Hausdorff spaces}
\defi{
A \textit{Hausdorff space} $\langle A,\mathcal{O}\rangle$ is a topological space, which satisfies that there for each $a,b\in A$ exist disjoint neighbourhoods $U,V$ around $a$ and $b$, respectively.\\
}

\theo{
Every metric space $\langle A,d\rangle$ induces a Hausdorff space $\langle A,d,\mathcal{O}\rangle$.
}
\textit{Proof.}
Let $\langle A,d\rangle$ be a metric space and $a,b\in A$ with $a\neq b$. Then by $(M_1)$, $d(a,b)\neq 0$. By Theorem \ref{metricspaceistopspace}, $\langle A,d,\mathcal{O}\rangle$ is a topological space where $\mathcal{O}$ is the topology generated from the open $\varepsilon$-balls. Construct two open $\varepsilon$-balls $B_a:=B_d(a,d(x,y)/2)$ and $B_b:=B_d(b,d(x,y)/2)$, which contain $a$ and $b$, respectively. If it is shown that $B_a$ and $B_b$ are disjoint, then we are done. Assume $c\in B_a$ and $c\in B_b$. Then $d(c,a)<d(a,b)/2$ and $d(c,b)<d(a,b)/2$. But then $d(c,a)+d(c,b)<d(a,b)$, which contradicts $(M_3)$. Thus is the induced topological space $\langle A,d,\mathcal{O}\rangle$ a Hausdorff space.$\qed$\\

\section{Manifolds}
\defi{
Let $\b{A}=\langle A,\mathcal{O}_A\rangle$ and $\b{B}=\langle B,\mathcal{O}_B\rangle$ be topological spaces. A map $f:A\to B$ is then called \textit{continuous} if it for every $S_B\in\mathcal{O}_B$ holds that $f^{-1}(S_B)\in\mathcal{O}_A$.\\
}

\defi{
A \textit{homeomorphism} $f:\b{A}\to \b{B}$ is a continuous map between topological spaces $\b{A}$ and $\b{B}$, that has a continuous inverse map. If such a map exists, $\b{A}$ and $\b{B}$ are said to be homeomorphic.\\
}

\pagebreak
\defi{
A \textit{local homeomorphism} $f:\b{A}\to\b{B}$ is a map between topological spaces $\b{A}=\langle A,\mathcal{O}_A\rangle$ and $\b{B}=\langle B,\mathcal{O}_B\rangle$, which satisfies that for each $a\in A$ exist a neighbourhood $S\in\mathcal{O}_A$, which satisfies $f(S)\in\mathcal{O}_B$ and that $f|_S:S\to f(S)$ is a homeomorphism. If such a map exists, $\b{A}$ and $\b{B}$ are said to be locally homeomorphic.\\
}

\defi{
An $n$-dimensional (real) \textit{manifold} $\langle A,\mathcal{O},\mathbb{R}^n\rangle$ is a second-countable Hausdorff space that is locally homeomorphic to a Euclidean space $\mathbb{R}^n$ by a collection of homeomorphisms $(\varphi_i)_{i\in I}$, called \textit{charts} - such a collection is called an \textit{atlas}.\\
}

\section{Smooth manifolds}
\defi{
Let $\langle A,\mathcal{O},\mathbb{R}^n\rangle$ be a manifold with atlas $(\varphi_i)_{i\in I}$. Then a \textit{transition map} is a map $\tau:\mathbb{R}^n\to\mathbb{R}^n$, given by
\eq{
\tau=\varphi_i\circ\varphi_j^{-1}\qquad, \varphi_i, \varphi_j\in(\varphi_i)_{i\in I}.
}
}

\defi{
A map $f$ is called \textit{smooth} if derivatives of all orders exist.\\
}

\defi{
A \textit{smooth manifold} $\langle A,\mathcal{O},\mathbb{R}^n\rangle$ is a manifold $\langle A,\mathcal{O},\mathbb{R}^n\rangle$, where all the transition maps in the associated atlas $(\varphi_i)_{i\in I}$ are smooth.\\
}

\section{Lie groups}
\defi{
A (real) \textit{Lie group} $\langle A,\mathcal{O},\mathbb{R}^n,\cdot,e,^{-1}\rangle$ is a smooth manifold $\langle A,\mathcal{O},\mathbb{R}^n\rangle$ and a group $\langle A,\cdot,e,^{-1}\rangle$ in which the operations $\cdot$ and $^{-1}$ are smooth maps.\\
}

\exam{
The \textit{general linear group} $GL(n,\mathbb{R}):=\langle A,\mathcal{O},\mathbb{R}^{n^2},\cdot, I_n, ^{-1}\rangle$ where $A\subset\mathbb{R}^{n^2}$ is the set of invertible $n\times n$ matrices with real entries, $\cdot$ is the matrix product and $I_n$ is the $n\times n$ identity matrix, is a Lie group.\\
}

\section{Disconnected spaces}
\defi{
A \textit{disconnected space} $\langle A,\mathcal{O}\rangle$ is a topological space in which there exist a collection of disjoint subspaces $(A_i)_{i\in I}$, which satisfies
\eq{
A=\bigcup_{i\in I}A_i.
}
}

\section{Totally disconnected spaces}
\defi{
A \textit{totally disconnected space} $\langle A,\mathcal{O}\rangle$ is a disconnected space, in which the connected subspaces are one-point sets.\\
}

\section{Compact spaces}
\defi{
A \textit{compact space} $\langle A,\mathcal{O}\rangle$ is a topological space in which each of its open covers has a finite subcover. E.g., for every collection $(U_i)_{i\in I}\subset\mathcal{O}$ with $A=\bigcup_{i\in I} U_i$ exists a finite set $J\subset I$, such that
\eq{
A=\bigcup_{i\in J}U_i.
}
}

\section{Boolean spaces}
\defi{
A \textit{Boolean space} $\langle A,\mathcal{O}\rangle$ is a totally disconnected compact Hausdorff space.\\
}

\section{Uniform spaces}
\defi{
A \textit{uniformity} $\phi$ on a set $A$ is a nonempty family of subsets of $A\times A$, that satisfies the following:
\begin{enumerate}
\item[($U_1$)] $\forall U\in\phi: \{(a,a)\mid a\in A\}\in U$
\item[($U_2$)] $\forall U\in\phi,V\subset A\times A: U\subset V\Rightarrow V\in \phi$
\item[($U_3$)] $\forall U,V\in\phi: U\cap V\in\phi$
\item[($U_4$)] $\forall U\in\phi\exists V\in\phi: (a,b),(b,c)\in V \Rightarrow (a,c)\in U$
\item[($U_5$)] $\forall U\in\phi: \{(b,a)\mid (a,b)\in U\}\in\phi$\\
\end{enumerate}
}

\defi{
A \textit{uniform space} $\langle A,\phi\rangle$ is a set $A$ equipped with a uniformity $\phi$.\\
}

\theo{
Every uniform space $\langle A,\phi\rangle$ induces a topological space $\langle A,\phi,\mathcal{O}\rangle$.
}
\textit{Proof.}
Let $\langle A,\phi\rangle$ be a uniform space. Define a subset $S\subset A$ to be open if and only if $\forall a\in S\exists U\in\phi:\{b\in A\mid (a,b)\in U\}\subset S$. Define $\mathcal{O}:=\{S\subset A\mid S\text{ open}\}$. It has to be shown that $\mathcal{O}$ is a topology.\\

$\b{(T_1)}$: It holds that $\emptyset\in\mathcal{O}$ trivially, since it doesn't have any elements. $A$ is open as well, by picking $U:=\{(a,a)\mid a\in A\}$, which is an element in $\phi$ by $(U_1)$.\\

$\b{(T_2)}$: Let $G_1,\hdots,G_n\in\mathcal{O}$. It has to be shown that $G_1\cap\hdots\cap G_n\in\mathcal{O}$. By definition there exist $U_1,\hdots,U_n\in\phi$ such that $\forall 1\leq i\leq n\forall a\in G_i: \{b\in A\mid (a,b)\in U_i\}\subset G_i$. It has to be shown that there exist $U_0\in\phi$ such that $\forall a\in G_1\cap\hdots\cap G_n: \{b\in A\mid (a,b)\in U_0\}\subset G_1\cap\hdots\cap G_n$. Define $U_x:=U_1\cap\hdots\cap U_n$. It thus has to be shown that $U_x=U_0$. Let $a_0\in G_1\cap\hdots\cap G_n$. Then by definition $a_0\in G_1\land\hdots\land a_0\in G_n$, meaning that $\forall 1\leq i\leq n: \{b\in A\mid (a_0,b)\in U_i\}\subset G_i$. But this means that $\{b\in A\mid (a_0,b)\in U_x\}\subset G_1\cap\hdots\cap G_n$. Since the choice of $a_0$ was arbitrary, it means that $U_x=U_0$.\\

$\b{(T_3)}$: Let $(A_i)_{i\in\mathbb{N}}\in\mathcal{O}$. It has to be shown that $A_0:=\bigcup_{i\in\mathbb{N}}A_i\in\mathcal{O}$. By definition, there exist $(U_i)_{i\in\mathbb{N}}\in\phi$ such that $\forall i\in\mathbb{N}\forall a\in A_i:\{b\in A\mid (a,b)\in U_i\}\subset A_i$. It has to be shown that there exist $U_0$ such that $\forall a\in A_0:\{b\in A\mid (a,b)\in U_0\}\subset A_0$. Define $U_x:=\bigcup_{i\in\mathbb{N}}U_i$. It has to be shown that $U_x=U_0$. Let $a_0\in A_0$. Then $\exists i\in\mathbb{N}: a_0\in A_i$ and thus $\{b\in A\mid (a_0,b)\in U_i\}\subset A_i\subset A_0$. Since this holds for all $i\in\mathbb{N}$, it is seen that $\{b\in A\mid (a,b)\in U_x\}\subset A_0$. Since the choice of $a_0$ was arbitrary, it means that $U_x=U_0$.$\qed$\\

\theo{
Every metric space $\langle A,d\rangle$ induces a uniform space $\langle A,d,\phi\rangle$.
}
\textit{Proof.}
Let $\langle A,d\rangle$ be a metric space. Define $\delta_\varepsilon:=\{(a,b)\in A\times A\mid d(a,b)<\varepsilon\}$ and $\phi:=\{\delta_\varepsilon\mid \varepsilon>0\}$. It has to be shown that $\phi$ is a uniformity on $A$.\\

$\b{(U_1)}$: Since $d(a,a)=0$ by $(M_1)$ and since $\varepsilon>0$, $(a,a)$ will be included in every $d_\varepsilon$ for all $a\in A$.\\

$\b{(U_2)}$: Let $V\subset A\times A$, $U\in\phi$ and $U\subset V$. It has to be shown that $V\in\phi$. Since $U\in\phi$, there exist an $\varepsilon>0$ such that $U=\delta_\varepsilon$. Since $U\subset V$ then $\delta_\varepsilon\subset V$. If $\max\{d(a,b)\mid (a,b)\in V\}<\varepsilon$, then trivially $V\in\phi$. If on the other hand $\max\{d(a,b)\mid (a,b)\in V\}\geq\varepsilon$, then set $\varepsilon_0:=\max\{d(a,b)\mid (a,b)\in V\}$, meaning $V=\delta_{\varepsilon_0}\Rightarrow V\in\phi$.\\

$\b{(U_3)}$: Let $U,V\in\phi$. Then there exist $\varepsilon_1>0$ and $\varepsilon_2>0$ such that $U=\delta_{\varepsilon_1}$ and $V=\delta_{\varepsilon_2}$. Then it is clear that $U\cap V=\delta_{\varepsilon_1}\cap\delta_{\varepsilon_2}:=\min\{\delta_{\varepsilon_1},\delta_{\varepsilon_2}\}\in\phi$.\\

$\b{(U_4)}$: Let $U\in\phi$. Then there exist an $\varepsilon>0$ such that $U=\delta_\varepsilon$. Define $V:=\delta_{\varepsilon/2}$ and let $(a,b),(b,c)\in V$. Then $d(a,b),d(b,c)<\varepsilon/2$ by definition, and then $d(a,c)\leq d(a,b)+d(b,c)<\varepsilon/2+\varepsilon/2=\varepsilon$ by $(M_3)$, meaning $d(a,c)<\varepsilon$ and then $(a,c)\in U$.\\

$\b{(U_5)}$ follows from $(M_2)$.$\qed$\\

\pagebreak
\section{Measurable spaces}
\defi{
A $\sigma$\textit{-algebra} $\mathcal{A}$ on a set $A$ is a family of subsets of $A$ with the following properties:
\begin{enumerate}
\item[($\Sigma_1$)] $A\in\mathcal{A}$
\item[($\Sigma_2$)] $S\in\mathcal{A}\Rightarrow S^c\in\mathcal{A}$
\item[($\Sigma_3$)] $(S_i)_{i\in\mathbb{N}}\subset\mathcal{A}\Rightarrow\bigcup_{i\in\mathbb{N}}S_i\in\mathcal{A}$\\
\end{enumerate}
}

\defi{
A \textit{measurable space} $\langle A,\mathcal{A}\rangle$ is a set $A$, equipped with a $\sigma$-algebra $\mathcal{A}$.\\
}

\rema{
Every measurable space doesn't induce a topological space, and vice versa.
}
\textit{Proof.}
That every measurable space can't induce a topological space is due to the fact that topologies requires closure of arbitrarily many unions $(T_3)$ and $\sigma$-algebras only required closure of countably many unions $(\Sigma_3)$. The reverse is true since $\sigma$-algebras require closure under complements $(\Sigma_2)$, and this is only always true for $X$ and $\emptyset$ in topologies $(T_1)$.$\qed$\\

\theo{
\label{measurablespaceismonoid}
Every measurable space $\langle A,\mathcal{A}\rangle$ induces monoids $\langle A,\mathcal{A},\cup,\emptyset\rangle$ and $\langle A,\mathcal{A},\cap,A\rangle$.
}
\textit{Proof.}
The proof follows analogously from Theorem \ref{topspaceismonoid}, since $A\in\mathcal{A}$ by $(\Sigma_1)$, $\emptyset\in\mathcal{A}$ by $(\Sigma_1)$ and $(\Sigma_2)$, $\cup$ is closed by $(\Sigma_3)$ and $\cap$ is closed by $(\Sigma_2)$ and $(\Sigma_3)$, since $a\cap b=(a^c\cup b^c)^c$.$\qed$\\

\theo{
\label{measurablespaceisboolalgebra}
Every measurable space $\langle A,\mathcal{A}\rangle$ induces a Boolean algebra $\langle A,\mathcal{A},\cup,\cap,\emptyset,A,^c\rangle$.
}
\textit{Proof.}
Let $\langle A,\mathcal{A}\rangle$ be a measurable space. Since $\b{A}$ is a monoid by Theorem \ref{measurablespaceismonoid}, as well as it is well known that both $\cup$ and $\cap$ are idempotent and commutative, both $\langle A,\mathcal{A},\cup,\emptyset\rangle$ and $\langle A,\mathcal{A},\cap,A\rangle$ are bounded semilattices. It has to be shown that $\langle A,\mathcal{A},\cup,\cap,\emptyset,A\rangle$ satisfies the absorption laws, distributive laws and that each element has a complement - ie. that for all $a,b,c\in \mathcal{A}$:
\begin{enumerate}
\item $a\cup (a\cap b)=a$
\item $a\cap(a\cup b)=a$
\item $a\cup(b\cap c)=(a\cup b)\cap(a\cup c)$
\item $a\cap(b\cup c)=(a\cap b)\cup(a\cap c)$
\item $\forall a\in A\exists a^c\in A: (a\cup a^c=A)\land (a\cap a^c=\emptyset)$\\
\end{enumerate}

$\b{(i)}$: Since $a\cap b\subset a$, it follows trivially. $\b{(ii)}$: Since $a\subset a\cup b$, it follows trivially. $\b{(iii)}$: Let $x\in a\cup(b\cap c)$. Then $(x\in a)\lor(x\in b\cap c)\lr (x\in a)\lor((x\in b)\land (x\in c))\lr ((x\in a)\lor(x\in b))\land((x\in a)\lor (x\in c))\lr (x\in a\cup b)\land(x\in a\cup c) \lor x\in (a\cup b)\cap(a\cup c)$. $\b{(iv)}$ follows analogously as (iii). $\b{(v)}$ follows from $(\Sigma_2)$.$\qed$\\

\section{Measure spaces}
\defi{
Let $\mathcal{A}$ be a $\sigma$-algebra defined on the set $A$. Then a \textit{measure} is a map $\mu:\mathcal{A}\to[0,\omega]$, satisfying:
\begin{enumerate}
\item[($\mu_1$)] $\mu(\emptyset)=0$
\item[($\mu_2$)] Given a family of pairwise disjoint sets $(S_i)_{i\in\mathbb{N}}\subset\mathcal{A}$, then
 \eq{
 \mu\left(\bigcup_{i\in\mathbb{N}}S_i\right)=\sum_{i\in\mathbb{N}}\mu(S_i)
 }
\end{enumerate}
}

\defi{
A \textit{measure space} $\langle A,\mathcal{A},\mu\rangle$ is a measurable space $\langle A,\mathcal{A}\rangle$ equipped with a measure $\mu$.\\
}