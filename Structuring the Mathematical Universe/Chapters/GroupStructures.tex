\chapter{Group structures}
\thispagestyle{fancy}

\defi{
A \textit{binary operation} on a set $X\neq\emptyset$ is a map $\cdot:X\times X\to X$, often denoted with infix notation $a\cdot b$ instead of the usual prefix $\cdot(a,b)$.\\
}

\section{Magmas}
\defi{
A \textit{magma} $\langle A,\cdot\rangle$ is a set $A$ along with a binary operation $\cdot$.\\
}

\section{Semigroups}
\defi{
A \textit{semigroup} $\langle A,\cdot\rangle$ is a magma satisfying $A\neq\emptyset$ and that $\cdot$ is associative:
\eq{
\forall a,b,c\in A: a(bc)=(ab)c.
}
}

\section{Monoids}
\defi{
A \textit{monoid} $\langle A,\cdot,e\rangle$ is a semigroup $\langle A,\cdot\rangle$ with an identity element $e$:
\eq{
\exists e\in A\forall a\in A: ae=ea=a.
}
}

\section{Quasigroups}
\defi{
A \textit{quasigroup} $\langle A,\cdot,\backslash,/\rangle$ is a magma $\langle A,\cdot\rangle$, satisfying:
\begin{enumerate}
\item $A\neq\emptyset$
\item $\forall a,b\in A\exists x,y\in A: ax=b\land ya=b$
\end{enumerate}
The unique solutions to these equations are $x=a\backslash b$ and $y=b/a$, called resp. left and right division.\\
}

\exam{
The nonzero rationals with division $\langle\mathbb{Q}\backslash\{0\},\div\rangle$ form a quasigroup $\langle\mathbb{Q}\backslash\{0\},\div,\backslash,/\rangle$ with $a\backslash b=b\div a$ and $a/b=a\div b$.\\
}

\section{Loops}
\defi{
A \textit{loop} $\langle A,\cdot,\backslash,/,e\rangle$ is a quasigroup $\langle A,\cdot,\backslash,/\rangle$ with an identity element $e$:
\eq{
\exists e\in A\forall x\in A: xe=ex=x.
}
}

\exam{
The nonzero \textit{octonions} $\mathbb{O}$, which are an extension of the quaternions $\mathbb{H}$ (which again are an extension of the complex numbers $\mathbb{C}$), form a nonassociative loop under multiplication - this loop is known as the \textit{Moufang loop}.\\
}

\section{Groups}
\defi{
A \textit{group} $\langle A,\cdot,e,^{-1}\rangle$ is a monoid $\langle A,\cdot,e\rangle$, in which every element $x$ has an inverse $x^{-1}$:
\eq{
\forall a\in G\exists a^{-1}\in G: aa^{-1}=a^{-1}a=e.
}
}

\theo{
Every group $\langle A,\cdot,e,^{-1}\rangle$ induces a loop $\langle A,\cdot,e,^{-1},\backslash,/\rangle$.
}
\textit{Proof.}
Let $\b{A}:=\langle A,\cdot,e,^{-1}\rangle$ be a group. Since every loop is a magma, it has to be shown that $\cdot$ has an identity element and that left and right division are well-defined on $\b{A}$. $\b{A}$ has an identity element since it is a monoid. For every $a,b\in A$, define $x:=a^{-1}b$ and $y:=ba^{-1}$. Then
\eq{
&ax=a(a^{-1}b)=(aa^{-1})b=eb=b\\
&ya=(ba^{-1})a=b(a^{-1}a)=be=b,
}

making \b{A} induce a loop.$\qed$\\

\section{Abelian groups}
\defi{
An \textit{Abelian group} $\langle A,\cdot,e,^{-1}\rangle$ is a group in which $\cdot$ is commutative:
\eq{
\forall a,b\in A: ab=ba.
}
}

\section{Bands}
\defi{
A \textit{band} $\langle A,\cdot\rangle$ is a semigroup in which every element is idempotent:
\eq{
\forall a\in A: aa=a.
}
}

\section{Semilattices}
\defi{
A \textit{semilattice} $\langle A,\cdot\rangle$ is a commutative band. The binary operation is often either $\land$, called ``meet", or $\lor$, called ``join".\\
}

\section{Bounded semilattices}
\defi{
A \textit{bounded semilattice} $\langle A,\cdot,e\rangle$ is a semilattice $\langle A,\cdot\rangle$ with an identity element $e$:
\eq{
\forall a\in A: ae=a.
}
}

\rema{
A bounded semilattice is equivalent to an idempotent commutative monoid.\\
}