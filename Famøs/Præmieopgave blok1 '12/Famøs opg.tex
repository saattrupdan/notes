% Document definition
\documentclass[a4paper,12pt]{article}

% Packages
\usepackage{fullpage}
\usepackage{amsmath}
\usepackage{amsfonts}
\usepackage{amssymb}
\usepackage{amsthm}
\usepackage{mathrsfs}
\usepackage{graphicx}
\usepackage[utf8x]{inputenc}
\usepackage{fancyhdr}
\usepackage{sectsty}

% Font size
%\sectionfont{\large}

% No indent
\setlength{\parindent}{0in}

% Theorem environment
\newtheorem{theorem}[section]{Sætning}
\newtheorem{proposition}[section]{Proposition}
\newtheorem{lemma}[section]{Lemma}
\newtheorem{corollary}[section]{Korollar}
\newtheorem{definition}[section]{Definition}
\newtheorem{remark}[section]{Remark}
\newtheorem{example}[section]{Example}

% User-defined commands
\renewcommand{\labelenumi}{(\roman{enumi}) }
\newcommand{\exam}[2][]{\begin{example}[#1]#2\end{example}}
\newcommand{\defi}[2][]{\begin{definition}[#1]#2\end{definition}}
\newcommand{\prop}[2][]{\begin{proposition}[#1]#2\end{proposition}}
\newcommand{\theo}[2][]{\begin{theorem}[#1]#2\end{theorem}}
\newcommand{\lemm}[2][]{\begin{lemma}[#1]#2\end{lemma}}
\newcommand{\coro}[2][]{\begin{corollary}[#1]#2\end{corollary}}
\newcommand{\rema}[2][]{\begin{remark}[#1]#2\end{remark}}
\renewcommand{\qed}{\hfill\blacksquare}
\newcommand{\qedeq}{\tag*{$\blacksquare$}}
\newcommand{\mrep}[3]{_{\mathscr #3}#1_{\mathscr #2}}
\newcommand{\lr}{\Leftrightarrow}
\newcommand{\sgn}{\text{sgn}}
\newcommand{\lcm}{\text{lcm}}
\newcommand{\vc}[1]{\underline{#1}}
\newcommand{\vto}[2]{\begin{pmatrix}#1\\#2\end{pmatrix}}
\newcommand{\mx}[1]{\underline{\underline{#1}}}
\newcommand{\mto}[4]{\begin{pmatrix} #1 & #2 \\ #3 & #4\end{pmatrix}}
\newcommand{\mtre}[9]{\begin{pmatrix} #1 & #2 & #3 \\ #4 & #5 & #6 \\ #7 & #8 & #9\end{pmatrix}}
\newcommand{\eq}[1]{\begin{align*} #1\\ \end{align*}}
\newcommand{\pic}[1]{\begin{center}\includegraphics[scale=.5]{#1.png}\\\end{center}}

% Header definition
\begin{document}
\noindent
\pagestyle{fancy}
\fancyhead[L]{Famøs opgave}
\fancyhead[R]{Dan '11 og Beatrix '11}
\setlength\headsep{15pt}

\subsection*{Problemet}
\textit{På en pind afmærkes to tilfældige punkter uafhængigt af hinanden, hvorefter pinden saves over på afmærkningerne. Hvad er sandsynligheden for, at de tre stykker kan samles til en trekant?}\\

\subsection*{Løsningen}
Først skal det klargøres, hvornår stykkerne rent faktisk udgør en trekant. Dette kommer til udtryk af følgende Lemma:
\begin{lemma}
\label{lemma1}
Tre linjestykker $a,b,c$ kan samles til en trekant hvis:
\eq{
|b-c|<a<b+c
}
\end{lemma}
\textit{Bevis.}
Det anses for velkendt at $a,b,c$ kan samles til en trekant hvis følgende uligheder gælder:
\eq{
a<b+c\qquad b<a+c\qquad c<a+b
}

De to sidste uligheder kan omskrives til:
\eq{
b<a+c\lr b-c<a\qquad c<a+b\lr b+c<a
}

Sammenlagt betyder dette netop at trekanten kan samles hvis $|b-c|<a<a+c$.$\qed$\\\\

Herefter kan sandsynlighedsfordelingen udregnes i følgende sætning:
\begin{theorem}
\label{saetning}
Et linjestykke med længde $l$ opsplittes i to stykker med længde $a$ og $d$, og efterfølgende opslitter $\max\{a,d\}$ i to stykker med længde $b$ og $c$. Da vil sandsynligheden $P(a)$ for at $a$, $b$ og $c$ kan samles til en trekant være fordelt efter funktionen:
\eq{
P(a,l)=\frac{a}{l}
}
\end{theorem}
\textit{Bevis.}
Man kan opfatte situationen som værende en problemstilling i sandsynlighedsmålrummet $([0,l],\mathcal{B}([0,l]),P)$. Først antages det at $a<l/2$ uden tab af generalitet da hvis $a>l/2$, kan $d$ og $a$ blot ombyttes, og hvis $a=1/2$ kan de tre stykker ikke blive en trekant pga. de skarpe uligheder i \ref{lemma1}. Da $a=1/2$ blot er en singleton og dermed er en $P$-nulmængde, kan det dermed antages at $a<l/2$.\\

Når snittet af $d$ skal foretages, skal de to resulterende linjestykker $b,c$ opfylde uligheden fra \ref{lemma1}:
\eq{
|b-c|<a
}

Betragt punktet $x:=(a+l)/2$, som er midtpunktet af $d$. Da må det andet snit maksimalt afvige $a/2$ fra dette punkt, og snittet må altså kun ligge i intervallet:
\eq{
I:=((a+l)/2-a/2,(a+l)/2+a/2)=(l/2,l/2+a)
}

Længden af intervallet er da $a$, og da vil $P(I)=P(a,l)=a/l$.$\qed$\\\\

Slutteligt kan det endelige resultat da formuleres i følgende korollar:
\begin{corollary}
Hvis et linjestykke opdeles i tre stykker, da er sandsynligheden $P$ for at de tre stykker kan samles til en trekant være:
\eq{
P=\frac{1}{4}
}
\end{corollary}
\textit{Bevis.}
Af Sætning \ref{saetning} følger det at sandsynlighedsfordelingen $P(a)$ for at de tre kan samles til en trekant, vil være $P(a)=a/l$, hvor $0<a<l/2$ per antagelse i beviset for Sætning \ref{saetning}. Da kan den samlede sandsynlighed udregnes som:
\eq{
P(l)=\frac{1}{l/2}\int_0^{l/2} \frac{a}{l}da=\frac{2}{l}\frac{1}{l}\int_0^{l/2}aDa(!)=\frac{2}{l}\frac{1}{l}\left[\frac{1}{2}a^2\right]_0^{l/2}=\frac{2}{l}\frac{1}{l}\frac{1}{2}\frac{l^2}{4}=\frac{1}{4}
}

Altså er den samlede sandsynlighed $1/4$.$\qed$\\\\

\subsection*{Ekstraopgaven}
Ekstraopgaven kan løses lignende, ved blot at se bort fra de korte stykke i beviset for Sætning \ref{saetning}, som dermed vil resultere i at $P(a,l)=a/(l-a)$. Dermed kan $P$ udregnes:
\eq{
P(l)=\frac{1}{l/2}\int_0^{l/2} \frac{a}{l-a}da=\frac{2}{l}\frac{l}{2}(\log(4)-1)=\log(4)-1
}

Dermed er $P$ i dette tilfælde $\log (4)-1$.

\end{document}